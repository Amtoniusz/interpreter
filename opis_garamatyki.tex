\documentclass{article}
\usepackage[MeX]{polski}
\usepackage[utf8]{inputenc}
\usepackage{amsmath}
\usepackage{amssymb}
\usepackage{mathtools}
\usepackage[makeroom]{cancel}
\usepackage{fancyhdr}
\usepackage{graphicx}
\usepackage{gensymb}
\usepackage{xfrac}
\usepackage{semantic}
\usepackage{proof}

\pagestyle{fancy}

\setcounter{MaxMatrixCols}{20}

\lhead{Antoni Burski \newline nr. ab394123}
\rhead{Deklaracja Języka} 

\begin{document}
% \hspace*{5ex}
\section{Wstęp}
Gramatyka będzie uproszczoną gramatyką języka C++.\newline
Typy zmiennych: int, bool, string , tuple<T> gdzie T to któryś z typów\newline
Funkcje: funkcje mają typ zwracany int, bool , string, tuple<T> lub void. Argumenty do funkcji są przekazywane przez wartość lub referencję. \newline
Pętla while, if else, wyrażenia arytmetyczne jak w C. \newline
Krotki: krotki tworzymy wyrzażeniem tuple (a1, ... an) i konstruktor ten nie może być cześcią wyrażenia, ale przypisanie krotce nowej wartości jest mozliwe\newline
komentarze zaczynają się od \# i trwaja do końca lini.\newline

\section{Gramatyka} 
Gramatyka wykorzystuje konstrukcje: Ident (identyfikator), String, Integer. \newline
Typy: \newline
$Bool \; = \; "true" \:|\: "false"  $\newline
Type\; = \; "int" \;\textbar\; "string" \;\textbar\; "bool" \;\textbar\; void \;\textbar\; tuple " \textless " Type " \textgreater "  \newline
Ref\; = \; "int\&" \;\textbar\; "string\&" \;\textbar\; "bool\&"  \;\textbar\; tuple\& "\textless" Type "\textgreater" ;\newline
Index \; = \; "[" Exp "]" \newline
IndexList \; = \; Index \{Index\} \newline
Var \; = \; Ident [IndexList] \newline
Operatory: \newline
$ AddOp \; = \; "+" \:|\: "-"  $\newline
$ MulOp \; = \; "*" \:|\: ":" \:|\: "\%"  $\newline
$ RelOp\; = \; "<" \:|\: ">" \:|\: "< =" \:|\: "> =" \:|\: "==" \:|\: "!=   "  $\newline
ListExp \; = \; [Exp] \;\textbar\; Exp \{"," Exp\} \newline
Exp1 \; = \; 
["("]\;Exp\;[")"] \;\textbar\;  
[{-\;\textbar\;!}] \;(  Var\;  \textbar \; \;Ident \; "(" \;ListExp \;")"\;  \textbar \;Integer \;\textbar \;String \;\textbar \;Bool \;)  \newline
Exp2 \; = \; [ "(" ] Exp1[ ")" ]   \textbar Exp2  MulOp  Exp2  \newline
Exp3 \; = \; [ "(" ] Exp2 [ ")" ]  \textbar Exp3  AddOp  Exp2  \newline
Exp4 \; = \; [ "(" ] Exp3 [ ")" ]  \textbar Exp4  RelOp  Exp3  \newline
Exp5 \; = \; [ "(" ] Exp4 [ ")" ]  \textbar Exp5  "\&\&" Exp4  \newline
Exp  \; = \; [ "(" ] Exp5 [ ")" ]  \textbar Exp  "\textbar\textbar" Exp5 \newline
\newline
NExp \; = \; Exp \; \textbar \; "(" NExp "," NExp \{"," NExp\} ")";\newline
Item \; = \; Ident [ "=" Exp ] \;\textbar\; Ident "=" "tuple" NExp \newline
ItemList \; = \; Item  \{ ","  Item \} \newline
Stm \; = \; 
Type ItemList \;\textbar\; 
Ref ItemList \;\textbar\; \newline
Var ("=", "+=", "-=", "*=", "/=", "\%=" ) Exp \;\textbar\;  Var = "tuple" Nexp \;\textbar\; \newline
"if" "(" Exp ")" (Block \textbar Stm) "else" (Block \textbar Stm) \textbar Exp ";" \;\textbar\;\newline
"while" "(" Exp ")" Block \;\textbar\; \newline
Type Ident "(" ArgList ")" (";" \textbar Block) \;\textbar\; 
"return" [ Exp ] ";" \newline
StmList \; = \;\{ Stm \} \newline
Block\; = \; "\{" StmList "\}" \newline
Arg \; = \; (Ref \textbar Type)  Ident \newline
ArgList \; =  [Arg] \;\textbar\; Arg \{ "," Arg \}  \newline

\section{Funkcyjność} 
  na 15 punktów:\newline
  01 (trzy typy) \newline
  02 (literały, arytmetyka, porównania) \newline
  03 (zmienne, przypisanie)\newline
  04 (print)\newline
  05 (while, if)\newline
  06 (funkcje lub procedury, rekurencja)\newline
  07 (przez zmienną / przez wartość / in/out)\newline
  na 20 punktów:\newline
  09 (przesłanianie i statyczne wiązanie)\newline
  10 (obsługa błędów wykonania)\newline
  11 (funkcje zwracające wartość)\newline
  na 30 punktów:\newline
  12 (4) (statyczne typowanie)\newline
  13 (2) (funkcje zagnieżdżone ze statycznym wiązaniem)\newline
  15 (2) (krotki z przypisaniem)\newline
  16 (1) (break, continue)\newline

Razem: 20+9 = 29

\section{Prykładowe programy}
\subsection{typy proste}
\begin{verbatim}
int a = 10; 
string b = "10";
bool c = "true";
int a1,a2 = 2;
string b1,b2 = "10";
bool c1,c2 = "true";
a1 = a+-a2;
c1 =  c2 || !c;
\end{verbatim}
\subsection{krotki}
\begin{verbatim}
tuple <int> a = tuple (1,2);
tuple<tuple<int>> b = tuple ((1,1),a,a);
a[2] = 3;
tuple<int> foo (tuple<int> a, tuple&<int> b){
    return tuple (a[1], 1);
}
foo(a,a);
a = tuple (2,3);
#komentarz
\end{verbatim}

\subsection{funkcje}
\begin{verbatim}
int counter;
void foo ();
void foo (){
    couter += 1;
}
int bar(int a){
    void foo(){
        couter -= 1;
    }
    foo();
    return a-1;
}
bar(1);
\end{verbatim}


\end{document}


